\section{Literature Review}

\begin{itemize}
    \item Core Models and Algorithms
    \begin{itemize}
        \item Source 1: "6 Restaurant Waitlist Metrics You Need to Track"         \cite{tableinRestaurantWaitlist}
        \begin{itemize}
            \item The information provided shows the average time customers are willing to wait to be sat as well as a model for "wait list turnover" to represent the rate of parties being sat to parties still on the wait-list queue.
            \item This information will be utilized in calculating customer patience levels and potential queue size. As customers wait beyond the expected "30 minutes" or less, patience levels will slowly plummet until they decide to abandon the system. The "Waitlist Turnover" rate will assist in employee utilization and speed of service metrics
            \item What I will adapt for my project specifically is the varying "acceptable wait time" that was surveyed for this report. Customers will have varying patience levels and that will be reflected in the initialized acceptable time the customer is willing to wait.
        \end{itemize}

        \item Source 2: "Ways to Reduce Restaurant Industry Food Waste Costs"
        \cite{foodWaste}
        \begin{itemize}
            \item From this article two things will be utilized. The statistic that "4\% to 10\% of food" produced never reaching customers, and how roughly "\$100,000 from each \$1 million" spent on food goes to waste.
            \item With the above statistics will be adapted into the human error equation that will affect speed of service. They will also reflect the increase to ingredient cost and further decrease ingredient availability.
            \item The human errors will increase that increase speed of service, and therefore decrease customer patience will range from the minimum 4\% to the maximum 10\% based on the size of the customer and food order queues compared to the number of available employees. This increase in error percentage will reflect on the ingredients/cost of ingredients. 
        \end{itemize}

        \item Source 3: "Food Services and Drinking Places: NAICS 722"
        \cite{blsFoodServices}
        \begin{itemize}
            \item The BLS has up to date information on the employee percentages within a restaurant. 
            \item This project will utilize the number of "Waiters and waitresses" and the number of "Cooks, restaurant."
            \item The above will generate a ratio of necessary employees and their occupation. As customer queues increase the necessity for more employees will as well, and this ratio provides information on what type of employee to hire for the shift.
        \end{itemize}

        \item Source 4: "How Many Employees Does It Take to Run a Restaurant?"
        \cite{escoffierManyEmployees}
        \begin{itemize}
            \item The model provided by Auguste Escoffier, School of Culinary Arts shows that each server will average at "five or six tables" each, as well as a total of "12-40 total employees" for each restaurant. 
            \item The information above will reflect employee availability and utilization rates, as well as providing a starting point for necessary employees to be hired.
            \item As servers reach the limit of "six tables" the chance of human error will also increase, and when looking at the data generated from the model, the total employee count provided will be apart of server to cook ratio.
        \end{itemize}

        \item Source 5: "Reducing Customer Wait Time at a Fast Food Restaurant on Campus"
        \cite{custArrivalTime}
        \begin{itemize}
            \item The model within the publication shows a bell-curve for customer arrival times and number of customers, starting from 9am to 5pm, which reflects one shift of a restaurant. 
            \item Each passing of the hour will have an influx of customers and will have customers leave the queue. The model provided will allow estimation of how many customers to expect per hour of the shift. 
            \item The total number of customers in the restaurant will be reduced from the provided model, as this reflects an college campus, however the ratio of customers per hour will be used to reflect queue times, employee utilization, etc throughout the day. 
        \end{itemize}


        
    \end{itemize}
\end{itemize}